Movement Composition Library

This tool allows you to \char`\"{}compose\char`\"{} complex 2d motion by chaining simple 2\+D movements together in configurable ways!

All your interaction will be with the Movement class. There are 5 types of simple movements, I will call these \char`\"{}primitives\char`\"{} \+:

\{Line, Curve, Circle, Wait, Sin\}

Complex movements are formed by chaining primitives. For all the methods listed below, there are more detailed usage notes near the function itself in the code.

There are also a bunch of examples under the Assets/\+Examples folder.

N\+O\+T\+E\+: \char`\"{}\+Teleportation\char`\"{} can be achieved by setting a wait\+Point in the W\+A\+I\+T primitive.

-\/-\/-\/-\/-\/-\/-\/-\/-\/-\/-\/--- C\+O\+N\+S\+T\+R\+U\+C\+T\+O\+R\+S

public Movement(\+Game\+Object entity);

public Movement(\+Game\+Object entity, string data);

public static Movement Init\+Movement\+From\+File(\+Game\+Object entity, string filename);

public static Movement Init\+Movement\+From\+Url(\+Game\+Object entity, string url);

-\/-\/-\/-\/-\/-\/-\/-\/-\/-\/-\/--- A\+D\+D M\+E\+T\+H\+O\+D\+S /// These methods allow you to add the aforementioned primitives to the current movement /// set. Primitives will be executed in the order in which they are added.

public void Add\+Line(\+Vector3 start, Vector3 end, float dur);

public void Add\+Sin(Vector3 start, Vector3 end, float dur, float amplitude, float freq, float phase=0);

public void Add\+Curve(Vector3 start, Vector3 end, float dur,Vector3 dep = default(\+Vector3));

public void Add\+Wait(float wait\+Time, Vector3 wait\+Point = default(\+Vector3));

public void Add\+Counter\+Clockwise\+Circle(\+Vector3 start,\+Vector3 center, float radians, float duration);

public void Add\+Clockwise\+Circle(\+Vector3 start, Vector3 center, float radians, float duration);

-\/-\/-\/-\/-\/-\/-\/-\/-\/-\/-\/--- C\+H\+A\+I\+N M\+E\+T\+H\+O\+D\+S /// The Chain methods are similar to the Add$\ast$ methods, but they start off /// where the previous primitive ends. Running Chain$\ast$ on an empty primitive /// set will cause an error.

public void Chain\+Line(\+Vector3 end, float dur);

public void Chain\+Wait(float dur);

public void Chain\+Sin(Vector3 end, float dur, float amplitude, float freq, float phase=0);

public void Chain\+Curve(Vector3 end, float dur,Vector3 dep = default(\+Vector3));

public void Chain\+Counter\+Clockwise\+Circle(\+Vector3 center, float radians, float duration);

public void Chain\+Clockwise\+Circle(\+Vector3 center, float radians, float duration);





// this allows you to shift the entire movement by the point shift public void Shift\+Movement\+By\+Point(\+Vector3 shift);

public void Start();

public void Set\+Repeat(int num = 0);

public void Toggle\+Trail();

// this is the game\+Object that will be instantiated on the screen as a trail marker. // It is prudent to give that gameobject some sort of timer so that it destroys // itself after a while. For testing purposes only. Once you've constructed your // movement, you should not ideally be using this trail. public void set\+Marker(\+Game\+Object m);

public void Update();

-\/-\/-\/-\/-\/-\/-\/-\/-\/-\/-\/--- A\+U\+X\+I\+L\+I\+A\+R\+Y M\+E\+T\+H\+O\+D\+S

// returns the state of the primitive at the specified index public object this\mbox{[}int i\mbox{]};

public void Save\+Movement\+To\+File(string filename);

public void Post\+Movement(string url, string movement\+Name);

public string Get\+Primitive\+As\+String(int index);

-\/-\/-\/-\/-\/-\/-\/-\/-\/-\/-\/--- O\+T\+H\+E\+R W\+E\+I\+R\+D T\+U\+N\+I\+N\+G A\+V\+A\+I\+L\+A\+B\+L\+E

// This is an experimental tuning feature that allows the user to add a // damping effect to the motion of the object by tuning the delta value. // However, slightly higher values make the frame rate look very choppy // and slightly lower than necessary values make the damping unnoticeable. // Furthermore, if the values aren't \char`\"{}just right\char`\"{} (and this is found by trying // a bunch of small values), then the movement begins to fail in all sorts of // funny ways.

// I\+F you can do a L\+O\+T of fiddling with deltas , then you can achieve // a decent looking damping effect on your motion.

public void Set\+Primitive\+Delta(int idx, float delta); 